%text
\usepackage{fontspec} %REQUIRES "xelatex" or "lualatex"
\usepackage{sourcesanspro}
\usepackage[T1]{fontenc}
\usefonttheme{serif}
\usefonttheme{professionalfonts}
\usepackage[tracking=true]{microtype}

\usepackage[brazilian]{babel}

\usepackage{multirow}
\usepackage{array}
%color
\usepackage{xcolor}
%scientific
\usepackage{siunitx}
\sisetup{
    output-decimal-marker = {,},   % Use comma for decimals (3,14)
    group-separator = {.},         % Use dot for thousands (1.000)
    group-minimum-digits = 4,      % Forces grouping for 1000 (standard is usually 5 digits)
    list-final-separator = { e },  % "1, 2 e 3" instead of "and"
    list-pair-separator = { e },   % "1 e 2"
    range-phrase = { a },          % "10 a 20" instead of "to"
    exponent-product = \cdot,      % Uses dot (3 . 10^3) instead of x
    inter-unit-product = \ensuremath{{}\cdot{}} % Dot between units (N.m)
}

%math

%code
\usepackage{listings}
\usepackage{algorithmicx}
\usepackage{algorithm}
\usepackage[noend]{algpseudocode}
\renewcommand{\lstlistingname}{Code}
% --- Frappe (Dark) ---
\definecolor{cFrappeBase}{HTML}{303446}
\definecolor{cFrappeText}{HTML}{C6D0F5}
\definecolor{cFrappeMauve}{HTML}{CA9EE6} % Keywords
\definecolor{cFrappeBlue}{HTML}{8CAAEE}  % Functions
\definecolor{cFrappeGreen}{HTML}{A6D189} % Strings
\definecolor{cFrappePeach}{HTML}{EF9F76} % Numbers/Booleans
\definecolor{cFrappeOverlay}{HTML}{737994} % Comments
\definecolor{cFrappeRed}{HTML}{E78284}   % Errors/Exceptions

% --- Latte (Light) ---
\definecolor{cLatteBase}{HTML}{EFF1F5}
\definecolor{cLatteText}{HTML}{4C4F69}
\definecolor{cLatteMauve}{HTML}{8839EF} % Keywords
\definecolor{cLatteBlue}{HTML}{1E66F5}  % Functions
\definecolor{cLatteGreen}{HTML}{40A02B} % Strings
\definecolor{cLattePeach}{HTML}{FE640B} % Numbers/Booleans
\definecolor{cLatteOverlay}{HTML}{9CA0B0} % Comments
\definecolor{cLatteRed}{HTML}{D20F39}   % Errors/Exceptions

% ==================================================================
% LSTDEFINES
% ==================================================================

% Style 1: Catppuccin Frappe (Dark Mode)
\lstdefinestyle{catppuccin-frappe}{
    language=Python,
    backgroundcolor=\color{cFrappeBase},
    basicstyle=\ttfamily\color{cFrappeText}\footnotesize,
    keywordstyle=\color{cFrappeMauve}\bfseries,
    stringstyle=\color{cFrappeGreen},
    commentstyle=\color{cFrappeOverlay}\itshape,
    identifierstyle=\color{cFrappeText},
    numberstyle=\color{cFrappeOverlay},
    % Specific identifiers for Python
    emph={self, cls, @classmethod, @staticmethod},
    emphstyle=\color{cFrappeRed}\itshape,
    % Function names (imperfect in listings, but helps)
    emph={[2]__init__, __str__, print, len, range},
    emphstyle={[2]\color{cFrappeBlue}},
    % Formatting
    frame=single,
    rulecolor=\color{cFrappeOverlay},
    numbers=left,
    numbersep=5pt,
    showstringspaces=false,
    upquote=true,
    breaklines=true,
    tabsize=4,
    captionpos=b
}

% Style 2: Catppuccin Latte (Light Mode)
\lstdefinestyle{catppuccin-latte}{
    language=Python,
    backgroundcolor=\color{cLatteBase},
    basicstyle=\ttfamily\color{cLatteText}\footnotesize,
    keywordstyle=\color{cLatteMauve}\bfseries,
    stringstyle=\color{cLatteGreen},
    commentstyle=\color{cLatteOverlay}\itshape,
    identifierstyle=\color{cLatteText},
    numberstyle=\color{cLatteOverlay},
    % Specific identifiers
    emph={self, cls, @classmethod, @staticmethod},
    emphstyle=\color{cLatteRed}\itshape,
    % Standard functions
    emph={[2]__init__, __str__, print, len, range},
    emphstyle={[2]\color{cLatteBlue}},
    % Formatting
    frame=single,
    rulecolor=\color{cLatteOverlay},
    numbers=left,
    numbersep=5pt,
    showstringspaces=false,
    upquote=true,
    breaklines=true,
    tabsize=4,
    captionpos=b
}
%graphic
\usepackage{tikz}
\usepackage[american, cute inductors]{circuitikz} % Circuitos
\usepackage{pgfplots}

\pgfplotsset{compat=1.18}

\pgfplotsset{
    /pgf/number format/.cd,
    use comma,
    1000 sep={.}
}

%macros
\usepackage{etoolbox}

%%%%%%%%%%%%%%%%%%%%%%%%%%%%%%%%%%%%%%%%%%%%%%%%%%
% Reference (biblatex) settings
%%%%%%%%%%%%%%%%%%%%%%%%%%%%%%%%%%%%%%%%%%%%%%%%%%
\usepackage[style=verbose,backend=biber]{biblatex}
\addbibresource{bibliography.bib}
\setbeamerfont{footnote}{size=\tiny}
\setbeamertemplate{bibliography item}{}% Remove reference icon.
\renewcommand*{\bibfont}{\footnotesize}

\usetheme{articlestyle}

%%%%%%%%%%%%%%%%%%%%%%%%%%%%%%%%%%%%%%%%%%%%%%%%%%
% Custom commands
%%%%%%%%%%%%%%%%%%%%%%%%%%%%%%%%%%%%%%%%%%%%%%%%%%

\newcommand{\trig}{\faAngleRight\hspace{1mm}}
\newcommand{\mycvitem}[4]{
    \begin{center}
        \begin{tabular}{p{0.15\textwidth} p{0.001\textwidth} !{\vrule width 2pt} p{0.001\textwidth} p{0.68\textwidth} } 
          \centering #1& & & \footnotesize{\textbf{#2}} \fontsize{8}{8}{\textcolor{invertedprimarycolor}{\textit{(#3)}}} \newline\fontsize{8}{8}{\trig #4} \normalsize\\ 
        \end{tabular}
    \end{center}
}
