\section{Introdução}

\begin{frame}{Cenário de exemplo}
  Imagine que temos um carro autônomo. Seu GPS quebrou. Tudo o que temos é um sensor de velocidade que lê a cada \qty{0.1}{\second} com precisão de \qty[per-mode = symbol]{0.2}{\metre\per\second}. Como é possível calcular a distância percorrida?
  \begin{figure}[H]
    \centering
    \includegraphics[width=0.5\textwidth]{demo/results/velocity_plot.png}
    \caption{Velocidade real do carro e dados capturados pelo sensor.}
    \label{fig:velocity_plot}
  \end{figure}
\end{frame}

\begin{frame}{Análise teórica}
  Lembre-se de que a velocidade é a derivada da posição:
  \begin{equation*}
    v(t) = \frac{dx}{dt}
  \end{equation*}

  Dessa forma, a distância é dada pela área abaixo da curva. Então, pelo \textbf{Teorema Fundamental do Cálculo}, obtemos: 

  \begin{equation*}
    \Delta x = \int_{t_0}^{t_f} v(t) dt
  \end{equation*}
\end{frame}

