\section{Introdução}

\begin{frame}{Cenário de exemplo}
  Imagine que temos um carro autônomo. Seu GPS quebrou. Tudo o que temos é um sensor de velocidade que lê a cada 0,1 segundos. Como é possível calcular a distância percorrida?
  \begin{figure}[H]
    \centering
    \includegraphics[width=0.5\textwidth]{demo/results/velocity_plot.png}
    \caption{Velocidade real do carro e dados capturados pelo sensor.}
    \label{fig:velocity_plot}
  \end{figure}
\end{frame}

\begin{frame}{Análise teórica}

  Math Connection:Recall that $v(t) = \frac{dx}{dt}$ (Derivative).Therefore, distance is the area under the velocity curve: $\Delta x = \int_{t_0}^{t_f} v(t) dt$ (Fundamental Theorem of Calculus).
\end{frame}

