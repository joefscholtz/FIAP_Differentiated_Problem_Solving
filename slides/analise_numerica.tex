\section{Análise Numérica}

\begin{frame}{Tempo Discreto}
  Diferente do que é considerado analiticamente, computadores trabalham com \textbf{passos discretos de tempo} para calcular resultados. Dessa forma, para que seja possível calcular o resultado desejado a partir de dados finitos, é necessário o uso de técnicas numéricas de integração que aproximam o resultado. A mais simples delas é a \textbf{Soma de Riemann}.
\end{frame}


\begin{frame}[fragile]{Integração numérica: Soma de Riemann}
  \begin{minipage}[c]{0.5\textwidth}
  \begin{figure}
    \begin{center}
      \resizebox{0.8\columnwidth}{!}{%
      \begin{tikzpicture}
        \begin{axis}[
            xmin=-0.5,xmax=6.8,
            ymin=-0.1,ymax=11,
            grid=both,
            grid style={line width=.2pt, draw=gray!25},
            major grid style={line width=.5pt,draw=gray!50},
            axis lines=middle,
            minor tick num=5,
            enlargelimits={abs=0.5},
            axis line style={latex-latex},
            ticklabel style={font=\small,fill=white},
            legend style={font=\small},
            legend cell align=left,
            legend pos=north east,
            declare function={
                f(\x) = 0.1*\x^3 - 0.5*\x^2 + 4;
            }
        ]
          \addplot[red, very thick, smooth, domain=0:6.5] {f(x)};
          \legend{$f(x) = 0.1x^3 - 0.5x^2 + 4$}
          \pgfplotsinvokeforeach{0, 1, 2, 3, 4, 5, 6}{
              \draw[fill=accentcolor, opacity=0.5, draw=accentcolor] (#1,0) rectangle (#1 + 1,{f(#1)});
          }
          \addplot[red, very thick, smooth, domain=0:6.5] {f(x)};
        \end{axis}
      \end{tikzpicture}
    }
    \end{center}
    \caption{Soma de Riemann à esquerda.}\label{fig:soma_riemann}
  \end{figure}
  \end{minipage}
  \begin{minipage}[c]{0.49\textwidth}
    Seja $f:[a,b]\rightarrow \mathbb{R}$ uma função definida no intervalo $[a,b]$ e $P = (x_0,x_1,\ldots,x_n) \in [a,b]$ uma partição do intervalo analisado tal que $a < x_0 < x_1 < \ldots < x_n < b$. A Soma de Riemann, $S$, aproxima a área abaixo da curva e é definida como:

    \begin{equation*}
      S = \sum_{i=1}^{n}f(\text{}^{k}x_i)(x_i-x_{i-1})
    \end{equation*}
    Com $\text{}^{k}x_i = x_{i-1}, \forall i$ definindo a Soma de Riemann à esquerda.
  \end{minipage}


\end{frame}
